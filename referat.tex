% !TEX encoding = UTF-8 Unicode
\documentclass[14pt]{article}

\usepackage{tempora}% for text
\usepackage{newtxmath}% for math
\usepackage[english, russian]{babel}
\usepackage[utf8]{inputenc}
\usepackage{times}
\usepackage{extsizes}
\usepackage[a4paper, left=20mm, right=10mm, top=20mm, bottom=20mm]{geometry}
% \geometry{landscape} % rotated page geometry

% See the ``Article customise'' template for come common customisations

\title{Основные уровни научного знания и механизм их взаимосвязи}
\author{Белянова Марина, группа ИУ5-12М}

%%% BEGIN DOCUMENT
\begin{document}
\maketitle
\tableofcontents

\section{Введение}

В данном реферате описываются и приводятся основные уровни, выделяющиеся в научном знании, а также различные механизмы их взаимосвязи. При написании реферата использовались труды С.А. Лебедева, Н.А. Губанова и др.
\subsection{Описание предметной области}
Концепция осмысления сущности научного знания и методов его получения, исследования науки и процедур научной деятельности, зародилась через науку об исследовании сознания, гносеологию, вопросы которой были центральной темой философии Нового времени. По прошествии некоторого количества времени, выделилась отдельное направление под названием философия науки, в рамках которой также изучается отношение мышления человека к существующему.


Одной из главных проблем, исследования науки является представление общей структуры научного знания. Традиционно принято выделять два больших уровня: эмпирический и теоретический. Любое научное знание является результатом деятельности рационального сознания, поскольку предполагает форму обсуждения семантики понятий.
\subsection{Научное знание}
Под научным знанием принято подразумевать знание, соответствующее определённым требованиям к ''научности``, среди них выделяется: предметность, однозначность, определённость, точность, системность, логическая доказательность, проверяемость, теоретическая и эмпирическая обоснованность, а также инструментальная полезность (применимость). Предполагается, что соблюдение этих свойств приводит к объективной истинности знания, поэтому часто ``научное знание'' отождествляют с ``объективно-истинным знанием''.

В частности, логическое и математическое научное знание должны удовлетворять более строгим требованиям научной рациональности: идеальная объектность, конструктивная однозначность, формальная доказательность, возможность аналитически подтвердить истинность научного знания, открытость для критики и опровержения, возможность улучшения и дополнения.

В книге ``Философия науки'' С.А. Лебедева приводится следующее определение научного знания: ``Научное знание -- знание, получаемое и фиксируемое специфическими научными методами и средствами (абстра­гирование, анализ, синтез, вывод, доказательство, идеализация, систематическое наблюдение, эксперимент, классификация, интерпретация, сформировавшийся в той или иной науке или области исследования ее особый язык и т. д.). Важнейшие виды и единицы научного знания: теории, дисциплины, области исследования (в том числе проблемные и междисциплинарные), области наук (физические, математические, исторические и т. д.), типы наук (логико-математические, естественно-научные, технико-технологические (инженерные), социальные, гуманитарные). Их носители организованы в соответствующие профессиональные сообщества и институты, фиксирующие и распространяющие научное знание в виде печатной продукции и компьютерных баз данных.
 \section{Уровни научного знания}
 \subsection{Чувственный уровень}
Формально чувственный уровень не принято относить к уровням научного знания, поскольку чувственное исследование объекта не может служить истиным источником научного знания, но стоит затронуть его, так как результат обработки продукта этого обследования сознанием становится научным знанием. К чувственному уровню относится проведение экспериментов, а также наблюдений. 

Чувственное познание, впрочем, является достаточно важным этапом в генерации научного знания, поскольку для того, чтобы узнать что-то новое о каком-либо объекте, в первую очередь необходимо провести некоторое чувственное взаимодействие с ним. После взаимодействия с объектом с помощью органов чувств, в сознании возникает образ объекта, которым можно оперировать, в частности, для создания какой-либо модели данного объекта и другого способа анализа окружающего мира сознанием. Чувственное познание обладает также определённой объективностью, поскольку физиология человека позволяет считать её общезначимой засчёт соблюдения у большинства людей биологических норм, влияющих на восприятие внешних объектов.
\subsection{Эмпирический уровень}
Для того, чтобы определить эмпирическое знание, необходимо разделить общую совокупность предметов на четыре типа: 1) ``вещи в себе'', 2) эти предметы, представленные в чувственном виде, 3) эмпирические или абстрактные объекты, 4) теоретические идеальные объекты. Подразумевается, что объекты второго типа проявляются не как результат чувственного восприятия, но как результат оного, спроецированный на сознание субъекта. При этом границы эмпирического сознания субъекта задаются в большинстве своём возможностями рассудка. На уровне эмпирического познания объекта формируется его абстрактная модель с помощью анализа содержания объекта, появившегося в сознании после анализа чувственного уровня.

Само эмпирическое познание -- это совокупность высказываний об эмпирических объектах, и как знание об объективной реальности оно может быть представлено только с помощью интерпретационного и идентификационного аппарата.  При этом, если считать эмпирическое знание результатом обобщения сознанием данных о наблюдениях или о проведении эксперимента, а также считать результаты наблюдений и экспериментов выводами из эмпирического знания, это будет ошибкой, поскольку считается, что между наблюдением и знанием существует отношение другого уровня: моделирование в одну сторону и интерпретация в другую.

Само же эмпирическое знание разделяется на несколько подуровней, каждый из которых логически может быть выведен из другого. В этом отличие подуровней эмприрического знания от уровней знания в целом. Так, из минимального уровня эмпирического знания, который представлен в виде единичных высказываний, называемых протокольными предложениями, индуктивным обобщением выводятся научные факты -- затрагивающие большую область применения утверждения, которые верны либо статистически, либо в общем случае. Эти утверждения определяют свойства и отношения субъекта познания и некоторые количественные характеристики. Важно отметить, что описанные выше первые два уровня неразрывно связаны и определяются какой-либо теорией, подтверждение или опровержение гипотезы которой и определяет цель эмпирического познания. 

Третьим элементом эмпирического уровня научного познания принято считать эмпирические законы. Эти законы, в отличие от научных фактов, имеют постоянную истинность во времени или в пространстве. Отмечается, что законы также имеют характер общих высказываний. Каждый из уровней эмпирического научного познания является результатом индуктивного обобщения, что подразумевает под собой неоднозначность выводов и может быть неверным, считается, что эмпирическое знание является гипотетическим, если оно не подтверждено.

Последний вид эмпирического научного знания -- это организованная система эмпирических законов, которая, впрочем, остаётся гипотетической и предположительной, поскольку опирается на предыдущие элементы эмпирического знания и лишь индуктивно выводится из них. 

Эмпирическое знание обобщает и структурирует чувственное знание, но результаты эмпирического познания не всегда являются истиной, поскольку для вывода его законов используется индукция, а содержание эмпирического знания уступает по полноте чувственному, поскольку использует лишь ту информацию о субъекте, которая необходима для формирования утверждений, фактов, законов.
\subsection{Теоретический уровень}
Этот предпоследний уровень знания с помощью построения абстрактных сущностей внутри себя выводит общие теоретические законы, верные для системы, построенной внутри разума. Источником теоретического знания может и не быть окружающий мир, например, для абстрактной алгебры или функций комплексных переменных, не существует отображения на реальную действительность, хотя данные области математики существуют. Они являются результатом деятельности разума, которая направлена не во внешний мир, а во внутрь себя, на развернутое собственное содержание сознания. 

Теоретическое знание можно считать знанием в себе и знанием для себя, которое формируется с помощью таких инструментов, как идеализация и интеллектуальная интуиция, с помощью которых создаются так называемые идеальные объекты, которых не существует в реальном мире (пример идеального объекта -- точка). Идеализация являет собой переход от свойств существующих объектов, наблюдаемых субъектом, к их крайним логически возможным значениям. 

При переходе от эмпирического объекта к идеальному происходит процесс, который удовлетворяет следующим свойствам: 1) исходный пункт всегда эмпирический объект и его свойства и связи, 2) для исходных данных происходит увеличение интенсивности свойства или связи и 3) происходит конструирование качественно нового объекта, свойства и связи которого принципиально не наблюдаемы, поскольку выведенный конструкт не существует в реальности. Такой объект назовём ``идеальным объектом первого рода''.

Существует ещё один способ, которым пользуются в основном в абстрактной математике для создания принципиально не существующих объектов и новых свойств, которые невозможно пронаблюдать даже в теории, это способ, при котором происходит введение идеального объекта путём задания его описания и свойств в идеальном мире. Назовём этот новый объект ``идеальным объектом второго рода''.
\subsection{Метатеоретический уровень}
\section{Механизмы взаимосвязи уровней научного знания}
\subsection{Взаимосвязь эмпирического и теоретического уровня}
\section{Заключение}
\end{document}