% !TEX encoding = UTF-8 Unicode

\documentclass[14pt]{article}
\usepackage[english, russian]{babel}
\usepackage[utf8]{inputenc}
\usepackage{extsizes}
\usepackage[a4paper, left=20mm, right=10mm, top=20mm, bottom=20mm]{geometry}
% \geometry{landscape} % rotated page geometry

% See the ``Article customise'' template for come common customisations

\title{Основные уровни научного знания и механизм их взаимосвязи}
\author{Белянова Марина, группа ИУ5-12М}

%%% BEGIN DOCUMENT
\begin{document}
\maketitle
\tableofcontents

\section{Введение}

В данном реферате описываются и приводятся основные уровни, выделяющиеся в научном знании, а также различные механизмы их взаимосвязи. При написании реферата использовались труды С.А. Лебедева, Н.А. Губанова и др.
\subsection{Описание предметной области}
Концепция осмысления сущности научного знания и методов его получения, исследования науки и процедур научной деятельности, зародилась через науку об исследовании сознания, гносеологию, вопросы которой были центральной темой философии Нового времени. По прошествии некоторого количества времени, выделилась отдельное направление под названием философия науки, в рамках которой также изучается отношение мышления человека к существующему.


Одной из главных проблем, исследования науки является представление общей структуры научного знания. Традиционно принято выделять два больших уровня: эмпирический и теоретический. Любое научное знание является результатом деятельности рационального сознания, поскольку предполагает форму обсуждения семантики понятий.
\subsection{Научное знание}
Под научным знанием принято подразумевать знание, соответствующее определённым требованиям к ''научности``, среди них выделяется: предметность, однозначность, определённость, точность, системность, логическая доказательность, проверяемость, теоретическая и эмпирическая обоснованность, а также инструментальная полезность (применимость). Предполагается, что соблюдение этих свойств приводит к объективной истинности знания, поэтому часто ``научное знание'' отождествляют с ``объективно-истинным знанием''.

 В книге ``Философия науки'' С.А. Лебедева приводится следующее определение научного знания: ``Научное знание -- знание, получаемое и фиксируемое специфическими научными методами и средствами (абстра­гирование, анализ, синтез, вывод, доказательство, идеализация, систематическое наблюдение, эксперимент, классификация, интерпретация, сформировавшийся в той или иной науке или области исследования ее особый язык и т. д.). Важнейшие виды и единицы научного знания: теории, дисциплины, области исследования (в том числе проблемные и междисциплинарные), области наук (физические, математические, исторические и т. д.), типы наук (логико-математические, естественно-научные, технико-технологические (инженерные), социальные, гуманитарные). Их носители организованы в соответствующие профессиональные сообщества и институты, фиксирующие и распространяющие научное знание в виде печатной продукции и компьютерных баз данных.
 \section{Уровни научного знания}
 \subsection{Чувственный уровень}
Формально чувственный уровень не принято относить к уровням научного знания, поскольку чувственное исследование объекта не может служить истиным источником научного знания, но стоит вскользь затронуть его, так как результат обработки продукта этого обследования сознанием становится научным знанием. К чувственному уровню относится проведение экспериментов, а также наблюдений. Чувственное познание, впрочем, является достаточно важным этапом в генерации научного знания, поскольку для того, чтобы узнать что-то новое о каком-либо объекте, в первую очередь необходимо провести некоторое чувственное взаимодействие с ним. После взаимодействия с объектом с помощью органов чувств, в сознании возникает образ объекта, которым можно оперировать, в частности, для создания какой-либо модели данного объекта и другого способа анализа окружающего мира сознанием.
\subsection{Эмпирический уровень}
Для того, чтобы определить эмпирическое знание, необходимо разделить общую совокупность предметов на три типа: 1) ``вещи в себе'', 2) эти предметы, представленные в чувственном виде, 3) эмпирические или абстрактные объекты. Подразумевается, что объекты второго типа проявляются не как результат чувственного восприятия, но как результат чувственного восприятия, спроецированный на сознание субъекта. 
\subsection{Теоретический уровень}
\subsection{Метатеоретический уровень}
\section{Механизмы взаимосвязи уровней научного знания}
\subsection{Взаимосвязь эмпирического и теоретического уровня}
\section{Заключение}
\end{document}